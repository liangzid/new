% Created 2020-05-28 周四 17:02
% Intended LaTeX compiler: xelatex
\documentclass[11pt]{article}
\usepackage[utf8]{inputenc}
\usepackage[T1]{fontenc}
\usepackage{graphicx}
\usepackage{grffile}
\usepackage{longtable}
\usepackage{wrapfig}
\usepackage{rotating}
\usepackage[normalem]{ulem}
\usepackage{amsmath}
\usepackage{textcomp}
\usepackage{amssymb}
\usepackage{capt-of}
\usepackage{hyperref}
\date{\today}
\title{}
\hypersetup{
 pdfauthor={},
 pdftitle={},
 pdfkeywords={},
 pdfsubject={},
 pdfcreator={Emacs 26.3 (Org mode 9.1.9)}, 
 pdflang={English}}
\begin{document}

\tableofcontents

今天下午身体不适,不爽。

\section{贾鹏学长  可达性检测}
\label{sec:org92fa8c0}
\subsection{background}
\label{sec:org8b71d02}
针对图数据

Label-Constrained Reachability Query

查询两个节点之间是否可以通过某些类别的边进行可达。在这里面,边的类别是需要进行考虑的。

\subsection{related works}
\label{sec:org3e96426}
1.BFS/DFS:很耗时间。精确查找
2.Index(通过对节点建立索引的方法)。
\href{./images/20200503193839.png}{Capture-20200503193839.png}
图中的就是索引。LI就是list。

\section{兰林学长 半监督学习图分类 ICLR2020}
\label{sec:orga2210c6}
\subsection{motivation}
\label{sec:org6f80107}
图卷积的方法1)在做半监督节点分类时过于普通(学习表示,之后分类)

2)相邻的节点未必有相同的label

\subsection{idea}
\label{sec:orgff32062}
思索查询节点和推理节点的相似性
\href{./images/20200503202612.png}{Capture-20200503202612.png}
\end{document}