% Created 2020-06-28 周日 12:12
% Intended LaTeX compiler: pdflatex
\documentclass[11pt]{article}
\usepackage[utf8]{inputenc}
\usepackage[T1]{fontenc}
\usepackage{graphicx}
\usepackage{grffile}
\usepackage{longtable}
\usepackage{wrapfig}
\usepackage{rotating}
\usepackage[normalem]{ulem}
\usepackage{amsmath}
\usepackage{textcomp}
\usepackage{amssymb}
\usepackage{capt-of}
\usepackage{hyperref}
\author{梁子}
\date{\today}
\title{存在与时间笔记}
\hypersetup{
 pdfauthor={梁子},
 pdftitle={存在与时间笔记},
 pdfkeywords={},
 pdfsubject={},
 pdfcreator={Emacs 26.3 (Org mode 9.1.9)}, 
 pdflang={English}}
\begin{document}

\maketitle
\tableofcontents


\section{前前言:关于一般文本的普遍特质与文学评论需有的基本精神}
\label{sec:org4c0fff5}

\subsection{引言}
\label{sec:org6a5d13e}
这几日在忙毕设答辩的事情,加上又要学出来一张驾驶证,东忙西忙里,读书变成了一种消遣的活动,因而那些需要很动脑子的书便搁置下来了。

今日父母在家,省下了我做饭的工夫,坐在书桌前无聊,又翻了翻上周在读的一本书,发现人的脑子总是会很快丢掉一些东西。出于一种留恋,我想把那些要忘掉的东西写下来,留待后来的自己耻笑。

我想先从《存在与时间》这本书开始。或许后来的很多带有“功用”性质的书籍都会获得这种待遇,甚至这类感悟会推广到自己读过的一些小说上。这些都说不准,因而只有先把手头的东西写一写了。

仍然需要声明:这些东西都是一些个人向的、暂时的想法,作者在谈及这些想法时不免是十分幼稚的,因此可以接受任何形式的探讨和批评,虽然很不爽。
\subsection{对一般文本的普遍特质的思索}
\label{sec:orgb50b801}
在进行对《存在与时间》这本书的讨论之前,我想先表明目前在我阅读一本书时进行评判的几个维度,以便为以后所有书籍的评判给出一个先天的探索角度。一篇文章,无论它是好是坏,里面都包含着某些“东西”。(这是废话。)文字作为人类语言的一种载体,它几乎可以承担一个人想象的所有内容。无论是一些故事,还是一些说理论道的东西,都可以囊括在里面。从这个角度出发,文本总是会包含这些特质:

1)文本必须是描述性质的。也就是,它必须做到“言之有物”,它必须讲某个东西。这个东西可以是真实的也可以是虚无的也可以是讲述文本本身,但是它必须要有。如果,某个人开发了一个软件,会从新华字典里随机选取一个字出来,这样不断地添加添加形成一篇文章,并随机地输入一些分隔的标点符号。这篇文章,这样的结果,可以算做是一篇文本吗?我的回答是,有可能可以。因为评判“字的组合”是否是一个文本的并不是这个文本如何产生,而是读者能否从中捕获到什么意义。从这个角度看,我对“一般文本”这个概念的定义可谓是特别主观了。

是的,我认同这种主观性。一个没学过中文的外国人,当他阅读我用中文写的一本非常下流的黄色小说时,他肯定没什么生理反应。这个例子不太恰当,还是拿画做比喻,某个抽象派大师画了一部非常厉害的神画,可是我并不懂如何欣赏绘画,所以我觉得这画和我幼儿园做的一些区别在何处。对于我而言,那幅画并不能算得上是一幅画,因为我“无可捕捉,无可沉浸”。很多时候文本也是会存在这一现象的。

幸运的是,文本在这一方面的进展显然没有像绘画那么极端,有人说可能是受限于语法,可是早在艾略特的现代诗时代,语法的松绑就开始有人跃跃欲试了,虽然效果一般。(突然想起来,现代诗算是文学赏析里一个独特的难点。)总之,文本必须言之有物,这个观点与其说是给出了文本的性质,不如说是定义了文本。

2)在描述性质的基础上,文本必须通过这样或者那样的方式展示它所描述的东西。这句话当然也是废话,并且“箭头”的方向还很可能倒置了。这个方面倒是促进了文学作为一种艺术的诞生。什么是艺术?术,是一种方法,一类套路的综合。而艺,是一个“巧妙的”、“有趣的”、带有一定“卖弄”意义的、让人眼前一亮的、观赏性强的、除非你练过否则一般人上手不行——这样一些“手段”的统称。所以,作为一种艺术的文学,就要在以上的那些形容词上有所展示。文学就是在这种背景下产生的,我要去表达一个东西,我要去描述一个东西,当一个人着重于他们表达的那些“东西”的时候,这个作品的灵魂在那些东西上,也就成了某些学科里的名著;当一个人确确实实表达了一些东西,但是将重点放在“我怎么去表达或者描述这些东西”的时候,他就开始思索文学。这样的作品,才能登堂入室成为“艺术”上的文学作品。

当然,这两个方向并不矛盾,论道性质的《论语》、《庄子》,文学欣赏起来也很不错,这当然和中国的哲学思想有关。西方虽然也存在很多这种“二者兼顾的东西(比如那个康德很喜欢的《爱弥儿》),但是他们的特质使得他们对这个东西分得很开很开,这是好事也是坏事,比如现在的我在写这篇文章,一篇除非你在思索这些观点否则毫无美感的文章,显然是收到西方的影响了。

在此处,我还要稍微歇息一下,讨论一下“如何描述”与“描述什么”之间的关联问题。很多人都不止一次地给我说,文学的价值不在于内容而在于形式,从而劝诫我少走弯路。这句话听起来总是有道理的。很早之前就有人说,故事就那么多类,早已被人讲尽啦。所以,当面对金庸、大仲马这些人的通俗小说时,人们总觉得他们对文学并无太大贡献。我自然理解这些观点,也无意为这些人辩驳。辩驳也是不必要的,人们就是喜欢那些故事,而不是那些“哲学精神”或者“手法”。此处想探究的,是形式与内容之间的关联。比如,给你一块肉(我们不去讨论这肉是猪肉还是羊肉)和一些佐料,让你做出一点食物来。这里的食材就很像是那些原料,而作家的高明之处就在于他们烹饪的方法。(当然,写作和做饭大有不同。这个类比同样不是特别合适。)烹饪的方法?他们烹饪的产物也许并不好吃(你看,这又是个主观问题),但是,如果他们发明了一种新的方法(比如烧烤的时候撒点孜然面),他们估计就会出名。但是烤出来的肉一定比红烧的好吃吗?这不一定啊,并且也没人在乎。形式和内容就是这样交融在一起的,很多评论的人都是在阅读文本时,尝试从里面挑出一些手法,然后进行分析。这是应该的,但是,更应该的是,你先沉浸在文字里,感受完这个故事,之后才能回过头来一点点思索。手法也是这样,某个方法不适用于所有的内容,某种方法以某种程度与某种内容结合,当这些量都恰当好处时,才能算完成了一个艺术品。在这里面,形式很重要,但是形式如何与内容结合,怎么结合,各种东西如何搭配,先放酱油还是先放醋,这些东西都很重要。比如某些小说里的感情,克制地写是一种味道,完全浸淫在里面又是另一种味道,你不能说这里面哪一种好,你只能说:你喜欢哪一种。

以下的情形都是比较低劣的,简单举个例子。假如有个人看着碟里的豆腐,还未动筷仅仅闻到了味道就破口大骂:“这也太臭了!为什么这么多人还爱吃?”假如这个人任凭别人夸赞其美味也毫不动心,只认同自己的鼻子,那么他的舌头又怎么知道臭豆腐的味道呢?同样地,还有一些人,他们总以为鸡蛋任谁煎都是鸡蛋,吃了媳妇炒的,便以为味道比全天下的大厨差不了太多。如果单纯是在一个鸡蛋上这么理论当然合适,但是,当食材越来越多,规模越来越大,事情就渐渐变了味道。当一个人读了一些简介、看了一些电视剧,就觉得自己已经品尝到长篇小说的滋味时,总有些东西同他说不清楚。当然,一个东西可以通过多种媒介表达,如果你想通过文字表达,并且你还不了解文字区别于影视等媒介的区别,那么是不是同样有些说不过去呢?总之,文本必须通过某种东西表现,当此处的“某种东西”成了主角,这样的文本就容易上升为艺术了。

这两点是我所认为的一般特质,当一个文本被归纳成某些类,就会多出一些特质,比如虚构类的具有一些特质,非虚构类的具有一些等等。(暂不去考虑这种分类是否武断。)由于本文是一个引言性质的文章,因此我想接着就“非虚构性”进行一种讨论。

\subsection{论“载道”的文的艺术性}
\label{sec:org9924b35}
先解一下题目。所谓“载道”的文,仅仅是说那些想表达一些思想的文章或书籍。比如,谈论某种哲学,谈论政治,谈论某领域的某个学说,或者发表对以上三种东西的一种评论。比如本文,虽然这个“道”不太有道理,但也是一篇“载道”的文章。中国古代,除了那些史书、诗词歌赋、小说,其他的东西多归于此类。在中国传统的文学里,这类东西没有一个不具有美感。这种风俗延续到了民国,比如闻一多等人的评论,仍然带着一股“文艺”的味道。而以我读西方的一些哲学书的感觉,他们在这方面显得更加纯粹一些,载道的就是载道的。后来,从民国到当代,我读到一些新的此类“载道”的文渐渐也失去了过去的那种艺术性,而彻底变成论道的东西了,这是好事也不是好事。

孔子对这种东西有过一个中庸的态度,或许可以作为一种思路:“质胜文则野,文胜质则史。”但是,鉴于目前社会里大家都不喜欢儒家,以及当代社会最喜欢激烈、走极端的样子,所谓“文质彬彬”的君子风怕是一去不复返啦。


\section{前言:哲学书的阅读与哲学笔记的撰写原则}
\label{sec:org174dd63}
很多人说,读海德格尔的书特别难,在我周围,有各种各样的朋友以各类方式暗示他们喜欢海德格尔,并且强调《存在与时间》一书的晦涩。这让我对此书产生了极大的兴趣,并为了“面子”、“炫耀”等类的东西开始了阅读。就我目前的感受是,该书并没有宣扬的那么难(至少在我读过的前五分之一处,并没有那么难),这本书虽然缺少文字上的美感(译者其实自己添加了一些美感),但是其行文逻辑毫无遮掩,句与句之间的关系也简直是明明白白,各种例子举起来更是不厌其烦。这样的一本书,如若读书的人不是那么焦躁,怎么可能读不下去呢?

诚然,读者此书中对很多概念的理解都是有深浅的,并且大多数人很多时候都觉得对里面的东西理解的不那么好,但是,我从读其原著本身来看,已经比读那些对其进行解释的公众号文章而获得的东西要多得多。因此,我认为读这一本书仍然是具有很大的意义的。

当然,虽然本文仅仅试图去记一个笔记,但是仍然有一些观点需要事先抛出来,比如面对一本哲学书籍(尤其是原著)时的态度问题。罗素曾经在《西哲史》里给出一种学习哲学家思想的方式。我们暂且不去讨论罗素本人有没有按照他所说的这种方式研究哲学然后写出那本《西哲史》,而是品味这种方式本身:首先,像在课堂上学习一门知识一样完全沉浸到那种思想里面,这种“沉浸”有点像“完全相信”、“完全由着书本的路子来”这种感觉,在这完成了之后,在自己觉得自己完全把捉了一个哲学家的思想之后,再采取另外一种方式进行“批判”,这种行为类似于牛羊等生物里面的“反刍(除,二声)”,也就是,把吃到肚子里的东西一点点通过回味消化成自己的营养。我认为这个学习思路是十分恰当,对于所有的非欣赏美的东西而言都十分恰当。而哲学是这类知识里面最危险的东西,如果一个人不能完全做到这一点,那么这人接受哲学思想其实具有极大的隐患。这种隐患体现在两个方面:

\begin{enumerate}
\item 如果不愿意“沉浸”,上来便批判。那么这个人将无法阅读中获得真正有用的东西,也读不懂书。更学不到书中试图令之领悟的思想。这类人就如同透过一个特定的望远镜看世界,自以为看的清清楚楚,但是,他不知道他所看见的所有的东西都是在一个特定的视角下看见的镜像而已。
\item 仅仅“沉浸”,并不反刍。这类人也很多,并且可能多是教科书读惯了之后养出来的毛病。其实这非常危险,因为哲学的思想,无论是西方的那些哲学家,还是中国的部分哲学家(比如庄子),其实从某种角度上都十分极端。如果一个人不是把这种思想吸收然后融合,而是直接读了一本哲学书, 然后就开始指东指西,那么其实非常危险。最怕的是这类人抛弃了远古时代的朴素哲学,直接看最近的哲学家的一些书籍,然后读了一两本之后便开始夸夸其谈,这样的危害更大。当然,我相信现实世界里这类人几乎不会存在。他们但凡多读基本这类的书,那些思想在他的脑子里打起架来,或者他愿意同周围的同学多辩论两嘴,问题也就自然而言不存在了。
\end{enumerate}

总之,这两类问题都是程度问题,一个人要做的就是时刻提醒自己调整好上述两点之间的比例关系。

下面再谈一点关于“哲学内容”所特有的问题。哲学内容所特有的问题是,你永远很难去在哲学里找到数学中的那种“纯粹的正确”。这样带来的问题是,任何哲学书籍都存在漏洞,并且,往往会出现这种漏洞直接把该哲学家建立的大厦全盘否认的情形。这样产生的问题是:如果一个人怀着学某种“正确”的哲学思想的想法去研究哲学,那么他就会特别绝望。我对哲学没有研究,但目前自认为的一种哲学研究的通路是:把哲学观念理解成一种角度,一种切入问题的方法。在研读哲学思想的时候,要有勇气和耐心阅读那些“错误”的哲学家,这样才能从中挖掘出金子。

读《存在与时间》亦如是,这类的哲学思考常常被海德格尔成为“路”,“小径”,这个比喻十分恰当。中国古代的所谓“道”,就是道路这种字面意思的延伸。

这都是一些废话,这类废话不吐不快,同时,也不致于让阅读下面笔记的人(如果有的话)觉得奇怪。当然,写这些废话也是为了让读这些笔记的人适应一下,一个人究竟可以话痨到何种程度。是啊,我又说了一段废话。


\section{导论:概述存在意义的问题之笔记}
\label{sec:org98f45cf}
再废话几句,在我看来《存在与时间》一书是极其学术的。
\subsection{存在问题的必要性}
\label{sec:orgd4fbcca}
存在的问题里,这个存在常常是指英文里的be动词,也就是am is are were was等东西,也就是中文里的“是”。从这个角度开始看,不得不说,存在是一个最基本的概念。
哲学所探究的东西都是非常基本的,比如个人生活在世界中,各种各样的目前科学无法解决的问题。因而,研究“存在”,并以之作为一个切入点,是非常基本且非常无聊的。

为什么说非常无聊?就像我在学习数学时不太在意数学分析中一些特别基本且显而易见的东西一样,我对海德格尔的“存在”一词也无太大的兴趣。但是,简简单单通过存在引申出的“对世界及周身人类的看法”,却是十分有趣的。并且,作者在研究“存在”问题时本身使用的方法论,也是非常有意思的(至少和目前大多数科学研究所使用的方法不同)。

先回到存在问题的必要性上,海德格尔认为存在具有三个特点:
\begin{enumerate}
\item 存在是最普遍的概念。这是当然,因为后续的任何概念都是在“是”的基础上定义的。因此,“是(也就是存在)”具有比其他一切概念更高的层级,也就是超乎一切族类之上的普遍性。而对于这样一种概念,大多数哲学家竟然无视它,也就是视之为“无规定性的直接性”,就不太合适了。
\item 存在不可被定义。因为任何定义的尝试都需要用到存在本身。
\item 存在是自明的。即在社会中生活的每个人都在语境中理解“存在”的意思。当然,这种可理解表明了先天的不可理解,也是海德格尔尝试解决的问题。对于这样一种自明的东西,海德格尔引用康德的语说“自明的东西且只有自明的东西——这些通常隐秘的理性判断,才是哲学家的事业”。
\end{enumerate}

就这样,书中给出了必要性。但是,在给出必要性的同时也发现,存在无法通过往常那类的“什么是存在?”来进行询问。因此,存在问题的形式结构也是需要讨论的。
\subsection{存在问题的形式结构}
\label{sec:org7c0d418}
在进行
\end{document}