% Created 2020-06-28 周日 19:16
% Intended LaTeX compiler: pdflatex
\documentclass[lang=cn]{elegantpaper}
\author{梁子}
\date{\textit{<2020-06-26 周五>}}
\title{十九岁的歌声}
\hypersetup{
 pdfauthor={梁子},
 pdftitle={十九岁的歌声},
 pdfkeywords={},
 pdfsubject={},
 pdfcreator={Emacs 26.3 (Org mode 9.1.9)}, 
 pdflang={English}}
\begin{document}

\maketitle
\tableofcontents




\section{第一章}
\label{sec:orgef9c603}

许久没来这里,没想到河床都干涸了。仍然是没有一丝泥土,大片大片破碎的石块。怪兀的灌木都伸着脖子向中间侵略,蝉鸣,密密麻麻的蝉鸣。

“爸爸,快跟上!”

他在叫我,我便加紧脚步。十年吗?似乎真的是,呵,真快,十年已经过去了。他还在往前走,穿着凉鞋,小心翼翼地踩在不平的碎石头上,缩着头试图绕过灌木枝,走入深处。那个声音越来越强烈,我又听到它了,如同天罗地网一样把我囚禁在这里,如此地热烈、聒噪,就像是肉体可感受到的电波,混乱着,像是祈盼着什么回应而又完全不在意的演奏。

那个声音让我无从躲避,我有些眩晕,头部发麻,眼前阵阵发黑,大脑里某些器件同其发生了共振。太阳左左右右晃动着,那天和今日毫无区别,都是死一样的寂静,被无视的大自然的合奏,未曾死掉的树上叶子的怪叫,以及,终其一生没有再听见的,十九岁末的歌声。

我永远无法把她描述出来,甚至,渐渐失去了捕捉记忆中跳跃的影子的直觉。它或许还潜藏在某个角落,当我突然地早醒时,当我在晃动的火车上听着歌无法入眠时,当在夏日,吹了空调躺在空空的床上时,它总会时不时跳出来。那时我似乎突然从一场大梦中惊醒,当我走在渐渐被当作“家”的城市里,听着行道树上零星的夏蝉在那里放肆地叫,我仍然抑制不住自己,再去挖掘那点零星的、早已面目全非的记忆。

我早已忘记什么时候认识的她。从小到大,我仍然觉得她仅仅有那一个样子。我记得她有黑得发亮的眼睛,却不记得它能打转。她仅仅是在那里微笑,半张着嘴巴,右手在空中挥舞,努力做着什么手势。我经常同她交谈,她更热爱倾听,然后尝试着问一点简单的问题。我几乎没有理解清楚她的问题,无论对错,她都给出正确的反馈。或许早些年,我和她更为亲密。那时我还能扯着她的手,或者直视她的脸庞。她的鼻梁很好看,其上一生都未承担过镜片的重压。时间越走越近,我却只能走在路上和她打个招呼,远远望着她,我怀念她的笑容。

她是光头。

很突兀吗?不,很自然。在我不记事的时候开始,我就觉得很自然了,仿佛她自然而然就是这样。熟悉我的人听到我的名字也不会奇怪,我熟悉她,所以我不奇怪她。我不愿意去回忆,每当想到那些日子,我的肋骨就莫名痛了起来。在记忆里,我并非什么好人。她经常受到欺负,从幼儿园到初中毕业。我从来没有帮助过她,我在同学耳中听到过无数有关于她的脏话,当时我也害怕的人甚至会往她的白色校服上滴墨水。我不和她同班,在学校里相遇,我都无法回应她的手势。

我甚至还能还原那些刻在我脑子里的话,或许是升旗时,我曾听见的叽叽喳喳。

“她竟然没有头发欸!”一个女生在那里用振幅很小但是每个人都听得见的声音说道。

“是啊!我听那谁说,她连腋毛也没有!”另一个人插话。

“那她那里会不会也没有毛……”

她就这样被孤立了,任凭那些风言风语传入她的耳朵。那个时候,孤立还是一件很可怕的事。人人都害怕特立独行,他们追求非主流的一切东西,因为只有这种非主流才能让他们进入主流,我也是这些人中的一员。

她曾经在学校里给我递过信,末尾写着歪歪扭扭的三个小字:“谢小蝉”。她写字很用力,纸张都会刻出痕迹来,但是字却偏小。每次我收到这些纸片,都会满不在意地装在裤兜里,等到回家后,在无人的时候细细阅读。

是的,我喜欢她。她或许从来不知道,或许也知道。我永远无法揣测这件事,在我二十岁时那个雨夜的邀请来看,她或许是知道的。但是,我拒绝了她的邀请,所以,我并不清楚那个邀请意味着什么。我更不清楚,这个邀请是否仅仅是针对我一个人。

我不确定,是不是如果我同意那次邀请的话,就不会发生那件事了。那个夏天,连同十九岁尽头的歌声一起,成为了我一生无法直视,却又频频回顾的梦。
\section{第二章}
\label{sec:org6dd1647}

我从小在农村长大,后来告别父母,去省内的一个城市生活。

我父亲在马路边经营着一家惨淡的小餐馆,我母亲在家里做工,后来工厂兴建,餐馆经营不下去,他俩便去做工资日结的零活。

我儿时喜欢爬树,那棵树竖立在一条石溪边上不知多少年了。我母亲曾经提起过它,它说在她出嫁来到这里时,那棵树就已经很老了。我和谢小蝉都称这棵树为尖叶树。不知多少次,我们都爬上这棵树,在茂密的树叶的掩护下,偷吃掉彼此的零食。从这个角度看,我和她至少是有一点友谊的。尖叶树的树干上有一个树洞,那个树洞被枝桠包围得很好,里面满是一些黑色的盐粒大小的东西。这些东西是什么,我至今也没有答案。我认为他们是树的粉末,谢小蝉则觉得那是死去的蚂蚁。

等到初中时,我便不再上树了。但是,她仍然乐于爬上那棵树,多是在盛夏,尖叶树长得繁茂的时候。后来,在我读大学时,她最后一次爬上了那棵树,她在树上做了从前从未做过的事情,她开始唱歌,唱了一个夏天的歌。

谢小蝉是一个哑巴。

我不知道这样描述是否准确,如果“哑巴”指的是那些无法说话的人,那么在第二十个生日之前,她确实没有发出任何有效的声音。

从后往前看,或许她在某个节点就已经可以发出声音了,然而她仍然可以维持她不说话的形象。这些事情都难以确定,本来我的记忆就不可靠,更别提建立在记忆之上的揣测了。如果说没有头发让她受到了欺负的话,无法说话让她更加难受。

听陈昊说,在学校里,每当一个新的老师点到她的名字,或者早读时她被发现不出声时,气愤的老师还未来得及开始责骂,便有人在旁解释一番。

“她说不出话来!”他们说道。

然后众人一片同情。

这些事情我不清楚,但是关于她的闲言碎语我却没少听。

“她不能说话,那会不会叫床啊!”“如果摊上这样一个媳妇,那么做起来到底有无聊……”

每当听到这类话,我都想握紧拳头给这些人来一下。但我从来没这样做过。我懦弱,我不想让别人觉得我很特殊,我不想让别人知道,我喜欢她,我喜欢这个没有一丝毛发的、不会说话的女生。

她似乎也渐渐察觉到问题所在了。她不会在学校里遇到我时跑过来,露出傻傻的笑,然后摆弄手势。她只是把我当作陌生人,没有特别的表情变化,看我一眼,然后就走开。而我,在这些地方,我从来没有主动和她打过招呼。

后来,等到我读高中时,和她交流的机会更少。她也不再写东西给我。更何况,在那时候,她开始谈恋爱了。
\section{第三章}
\label{sec:org8d78f39}

真快,按照这种流水账,已经到高中了。前面我从来没有提及过她的爱情。因为我嫉妒,因而故意美化了我的记忆。我仍然很羡慕她的爱情,尽管从某种程度上,那个男生算是我的情敌。

那个男生就是陈昊。甚至,他和她不是因同学关系而相认。实际上,我们三个人是从小一起玩到大的。你看,我在前面描述的时候,故意略去了这样一个人。而实际上,尖叶树是我们三个人起的名字,我们三个人一起爬树,甚至我还是那个爬得最慢的。

“上来吧,没关系的!”一开始,他经常这样鼓励我。

“我——我在下面就好了!”

“这上面可有一个树洞!可以存放物资的哦!”

总之,陈昊出现在我童年的方方面面,甚至,他和谢小蝉的关系还要更近一些。和我不一样,他有勇气去捍卫他朋友的尊严,而我仅仅是在背后安慰她而已。或许就是因为他的帮忙,各种各样的流言都针对他生长起来了。

“为什么每次咱们的陈昊(这群女生总是这个叫法)都替谢小蝉说话呀?”

“因为咱们的陈昊有爱心。”

“我觉得没这么简单,他该不会爱上那个不会说话、身上没有一根毛的人了吧!”

“有可能,听说他们可是青梅竹马……”

陈昊倒是不在意这些事情的。他比我们大一岁,也一直是我们三个人的头儿。学习成绩一直数一数二,喜欢打篮球,长得很高,也很会和别人打关系。他从不会被班级孤立,所有的女生都向他示好,而所有的男生都想做他兄弟。我把他看作是最好的朋友。唯一可惜的是,他也喜欢她。

我是在高中知道这件事的,他们或许也是高中才开始的吧。也有可能更早,大家在树上吃零食的时候,或许他就对她更好一些。这些我都不清楚,回溯过去也是一件极其艰难的事。谢小蝉没有读高中,毕业之后她就在附近的工厂打工了。

那座工厂横亘在村外的公路边,每天都会发出持续不断地捣击声,每个经由那里走入农田的人,都会有一种奇异的不安。谢小蝉的工作是把袋子放在一台快速运行的缝纫机口上,然后拿下来。他们两个人的爱情或许是从骑电动车开始的,当然,我这么说可能是因为我仅仅看到了他们在一起骑电动车。

一切的起因都是,我发现陈昊的电动车会莫名地出现在厂子门口。后来我才搞明白,原来他和谢小蝉会一起去学校,然后她再骑着车回来。夜晚,在晚自习结束后,她则过来接他。叙述这些事情并不是多么让人舒服的事,我第一次发现这个流程时,偷偷跑到那棵树边想着大哭一场。很可惜,我再怎么压抑自己的心情,也没流出来眼泪。后来,我就和他们一起上下学了。

我高中的成绩很差,尤其是理工科,函数求导、受力分析什么的都看不懂。但我仍然对生物感兴趣,尤其是讨论遗传的部分。陈昊的成绩永远比我要好很多,分科转了理,去了学校里唯一的实验班,最后他参加高考,去了首都的一所重点大学。后来我一直在想,如果他的成绩差一些,或许他就会和谢小蝉在一起了。但是,他又是那么的优秀,优秀到在我一个情敌性质的旁观者视角来看,他都过于完美。我永远无法想清楚这件事,如果我能想清楚,我的高考也会很好。

刚刚的话说了一半,我对生物很感兴趣,但我只对高一的生物感兴趣,因为我后来便转去学文了。我仍然记得高中生物课本上关于一个外国人在寺庙里种植豌豆的问题。那个外国人每做一次实验,就要花费一整年的时间等待结果。而他所作的一切,不过是在探究这些豌豆到底是什么品种,是高的还是矮的。我也想说服我父亲进行一次这样的实验,但是不知什么原因,这项计划被搁置了。这么多年过去,我成了家,成了丈夫和父亲,这项实验还是没有进行。

我对Dd,dd这类东西很感兴趣,尤让我记忆深刻的是关于“秃头”的基因问题。我关注这个问题可能是因为班主任的地中海,但我更相信,是因为谢小蝉。那道题或许是在某次令人头疼的考试中出现的,大致的意思是,只有父母双方都带有少见的隐性基因且这种隐性基因以很小的概率结合在一起成为一个纯合子,女性才会变成光头。而男性,只要只要带有一个这样的基因就几乎免不了这样的命运了。

我或许可以考证谢小蝉是不是这种情况,如果是,那么她父亲肯定是秃头,而她母亲,或许是(但更大概率不是)。分析到了这里就无法进行下去了,因为她没有父亲和母亲。

\section{第四章}
\label{sec:org52908bb}

在世界上生活了这么多年,我有一种冥冥中的直觉:如果一个人无因果地发生了悲剧的事,那么悲剧将在她身上接二连三地显现。

谢小蝉的家在村子深处,也就是尖叶树所在的地方。她的爷爷是清扫公路的环卫工人,而她就是在垃圾桶里捡来的。我父母给我讲这个时都会露出同情的表情,虽然他们的同情也就仅限于此了。我不知道为什么她的父母会抛弃她。是因为她刚出生就不会哭泣,所以被认为是一个哑巴吗?还是说,她连最微弱的汗毛都没有?以前我总是抱有这种想法,但是现在,当我听过她在夏日的歌唱之后,我才觉得事情或许没有那么简单。

总之,她的爷爷收养了她,还给她起了一个好听的名字。听父母说,她的爷爷是抗美援朝下来的老兵,媳妇难产,最终膝下无人,便一个人过着。他或许会打枪,这我不确定,但是我确信他不会写字,这也是为什么她名字中的“婵”变成了“蝉”。

她爷爷后来死了,她那时候刚刚成年不久。不到半年后,她也死了。她爷爷去世时,仅仅是她一人,加上几个街坊邻居帮衬着合了葬,虽然还吹了唢呐,但没有任何的场面。在她死去时,村里的人、县里的人、从全世界汇集的无数的名流、专家为她送行。然后用红砖头,白灰粉,连同最廉价的黄土一起为她修了坟。她仍然是孤寂的,她的死和她爷爷本身没有分别。人们在修完坟,填完土,种完松树之后便把她忘了,她就被忘在那里了。

后来,和她谈过恋爱的人,娶了妻子,做了教授,成了上流的体面人。喜欢过她的人,也娶了妻子,找了座陌生的城市,过着非富非贵也不是赤贫的生活。我们都走了,离开了尖叶树和那个村庄,只有她留在那里。

\section{第五章}
\label{sec:org999a061}

不知道有多少人注意过鸣蝉。大多数想到这个东西,或许就是儿时抓它们的时候,或者是人过中年到了耳朵开始自己发出声音的年纪。

我曾经听过一个故事,或许也是谣言。它们说蝉会在夏夜(尤其是下雨的夏夜)从地里冒出,爬到树上,抓紧树皮,进行蜕皮。之后,雄蝉会不停地鸣叫,以期待完成交配。交配产生的卵会通过雨水或者落叶的方式归于大地。这些卵将在地底下蛰伏许久,两年,三年,五年,七年,十一年,十三年,十七年,最长可以到十九年。

“蝉蛰伏的年份都是一个质数。这有利于躲避天敌的风险。”它们这么说。我不懂数学,也不能理解这里的原理。我只知道,人类似乎是不需要蛰伏的,人类会庆祝满月,庆祝十八岁、二十岁的生日,庆祝六十大寿、八十大寿。没有人去寻找质数。大家也不喜欢蛰伏。

在我还在家乡生活的时候,走在路上,每个夏天都听得见密密麻麻的蝉鸣。如果说青蛙的叫声还算有些调子,那么蝉鸣就是绝对的单调、贫乏,绝无任何欣赏的价值。后来,我经由高考去东北的一所大学念书,那里几乎听不见蝉鸣。即使有,也相当微弱,或许是因为不够炎热吧。每当这个时候,我都会想起来一个光着头的、露着笑脸,去编织厂打工的女孩。她在那里做着手势,或者穿着发白的校服,递给我一页纸。

蝉喜欢有水的地方,这也不足为怪。溪水边的树上总是有着最热烈的蝉鸣。尖叶树旁边曾经有一条石溪,这条石溪下面是不定大小的碎石,却没有任何的一点淤泥。溪边生长着大片的灌木——他们很可能是尖叶树的后代。在那些长歪了的小树上,布满了密密麻麻的蝉。我少年时经常在里面洗澡,把从园子中偷来的桃子,菜地里偷来的黄瓜、西红柿,商店里买来的辣条都丢在里面,然后仰浮在水里听蝉的声音。那时候的陈昊喜欢跳水,从桥头上往下落,然后激起一大片水花。我不喜欢那些过于剧烈的运动。我就想安静地躺在水里,任水流带着我往下走,我把耳朵放进河水下面,这时整个世界的声音都会很奇怪,包括蝉的鸣。水拐了一个弯之后就越来越缓,下游有一段规整的河口,岸边尽是洗衣服的女人。从十几岁,到八九十岁。谢小蝉也会在那里洗衣服。有时她会帮我把在上游的衣服拿到河口前面,我穿上后便赤着脚跑回去,然后继续落入水中,仰着头漂下来。我很感激她能帮我拿衣服,而不是像陈昊一样藏在某个地方。

是的,我不逆游。朝上游的方向用力扑打水面是一件很累人的事。或许从那时候开始,我就意识到我这一生是在随波逐流了吧。

我初中毕业之后便不再下河洗澡。陈昊读高中时,假日的时间完全交给了辅导班和刷题。我和他见面的场合也越来越少了。而我呢,我读一些没用的书,或者把自己锁在房间里,思索所谓的创作。

“爸爸,我找到水了!这里有一滩水!”他在叫我。

他正蹲在一小摊水前,朝水里好奇地张望,“这水很清呀!石头也特别干净!”他当然不懂,这代表石头很早之前也是露着的,而积水只不过是雨水又重新汇入了干涸的底。

“这个水里会不会有鱼啊!”他卷起裤脚,小心地将凉鞋放到水里的石头上,又朝前走过去了。

\section{第六章}
\label{sec:org4f62ea8}

那个难忘的夏天应该位于大一的暑假,或许是大二,我也记不清楚了。我只知道,那年的雨水很多,早在阳历六月份就进入了雨季。当然,太阳也是出奇的毒辣,土地里几乎所有的蝉都在一次又一次的雨水中露出头来,凑成一首巨大的合唱曲。那一年,所有的人都对气候变暖深信不疑。

当我应付完学校的期末考试,乘着慢腾腾的火车回到家乡时,父亲已经老了。厂里裁人,他们便失了业。他又回到了那个小餐馆,母亲则着手去校门口卖早餐。

我得知的另一个消息是,谢小蝉的爷爷去世了。

我和谢小蝉已有半年没见面。她还是光着头,只不过皮肤晒黑了很多。那张让我不敢直视的脸还在那里。有时走在路上,我看得见她的微笑。我曾经在网络上和她聊过天。她对她爷爷去世的事只字未提。那可能还是早些时间,她问我能不能再给她推荐几本书。

“我记得你曾经讲过一本书,里面的主人公喜欢爬树。他在树上过日子,是不是?”

“不是喜欢爬树。而是他爬上树后,从来没有回到地面。”

“吃喝拉撒都在树上?”

“是啊。不行么?”

“那这样的一个人,能活多久呢?”

“反正那个主人公活得不算短。”我那时候似乎很忙,也可能是心情很糟。不然面对她,我不会没有什么话要讲的。后来我把那本书寄给了她,让她自己去看,再后来,我收到了这本书。我再也没有读过它,《树上的男爵》。

当时的她,似乎说过:“每当想起这个故事,我就会想起尖叶树。”

尖叶树自然可以称得上是一棵大树,但是它孤零零地矗在那里,距离森林却差得远呢。

那个夏天,谢小蝉失去了她唯一的亲人,我什么也没有失去,什么也没有获得,而陈昊,则带回了他的女朋友。

我不知道这对谢小蝉而言算不算得上是一种刺激,更不知道是不是这种刺激导致了后来发生的所有事。我当然倾向于不是,虽然每次理性翻过身来都会告诉我,大概这就是真相。或许这真的就是真相,亲人去世,爱人离开,她便做那件事去了。

我自然不知道陈昊什么时候在大学交了女朋友,正如我不知道他什么时候和谢小蝉提的分手。记忆模糊了之后,内心不自觉的美化,我都开始倾向于他俩根本没有谈恋爱。他们的分手,在我看来,或许是不可避免的。一个初中毕业生,与一个将来不知道会读到什么学位的人才,更何况那位初中毕业生还说不出话,还没有头发。

可是尽管如此,我仍然因此仇视陈昊。我们虽然几乎早就断了交流,但我渐渐讨厌起他了。可是我也知道,他是无辜的。他在高中时就因为和谢小蝉谈恋爱而被父母百般责骂。他或早或晚一定会放手的。

我不知道谢小蝉会怎么想,估计分手的时候,她也无法发言说些什么。即使是面对面,她也发不出什么声音,更何况在电话里,她永远无法吵架,她或许打了一些什么字。

时隔多年,她又选择爬上尖叶树。她邀请过我,我不知道她有没有邀请陈昊,但从结果上看,即使她邀请了他,他也没有过去。那天晚上的雨下得很大,据说是有台风。

\section{第七章}
\label{sec:orgeb0ba04}

我所居住的小村庄,在山东东南部的一个偏僻角落里。对于这个地方,台风还是一个稀奇的东西,所以提前两天,大家就迫不及待想要看看台风什么样子了。台风对我父亲的生意而言不是一件好事,但在平时他的生意就很惨淡,因此也不算什么大问题。

记得那晚的台风刮断了几根电线,刮断了几根网线,让几个家庭失了电,断了网,因而失去了发朋友圈的最好机会。几棵招风的树栽倒了下来,玉米军队整体倾斜了30度。这都不是什么令人关心的事,不出一夜,电与网都恢复了。至于庄稼,大多数人都失去了农业社会的那种心疼。

接着出现的就是河水暴涨,石溪完全成了洪水倾泻的牙口,听说好几个鱼塘被冲破,村民都在那里拿着网捞鱼,无论是用来吃的鲤鱼,还是用来看的金鱼。金鱼多是不好养的,没有打氧机,常常一夜就窒息死去了。

可是台风的威力也就持续了一夜。甚至,仅仅是前半夜。早晨,太阳又出来了,人们还没来得及去适应大雨带来的降温,又要妥协于这种炎热。很快,河水消下去了,土地变干了,空气变得潮湿而闷热,令人淌不出汗。

我每天都在空调的庇护下颓废度日,绝不敢轻易出门。

谢小蝉在树上的事,还是在吃午饭时从父母口中得知的。

“她爬到那棵树上去干嘛?那棵树上也没什么好东西呀。”母亲觉得奇怪。那棵树上什么也没有,只有不存在了的记忆。

“我哪里知道,据说她是刮台风那晚上上去的。那么大的风,竟然抱着树在上面过了一整晚!”父亲也把它当作怪谈。“害!大伙在树下请她下来,她也没什么表示,要不是谢家兄弟送上去点吃的,她就挨了一天的饿了!”

至于陈昊那边是什么反映,我却不知道。

第二天,天气越发的燥热。我得知了那个消息,她开始说话了。

准确一点,她并没有开始说话,她只是在那里哼歌。没有人知道她是哼给谁听的,也没有人能从里面听出来什么歌词。我听过无数首歌,琴箫鼓瑟,钢琴、小提琴,各类的民歌,嘻哈,现代歌曲。她的歌声我找不到一个归类。

那是一种痴迷。村里所有的闲人,小孩,都聚集在那棵大树下,他们从来没有听过这样的声音。如此轻和,柔软,包含着所有的感情,却又唱得虚无缥缈,像是仙人摆动的拂尘。我能从歌声里听出执拗,但是这种执拗却一点都不艰涩,就像是无时无刻不向下流动的溪水,里面没有任何的委屈,仿佛世间一切一切的事,都变得毫无重要性可言。

“老天爷,我要死了吧!”有人甚至跪在了那棵树下。她的歌声可以感染一切。

尖叶树下,人越聚越多,从几十年前集会放电影之后,从来没有这样一种将所有人聚在一起的场面。有人把视频录了下来,发到网上。村里的书记开始向县里汇报。这件事很快上了热搜。从来没有人听到过这么美的声音,两三天后,大批的人从远方坐着火车赶来。

谢小蝉的声音混杂在聒噪吵人的蝉鸣里,蝉鸣让人的耳朵觉得炸裂,但是她的歌声却又能让人在灵魂最深处伤口悄悄愈合。

电视上收视率最高的唱歌节目在这个暑假也失去了它的魅力,无奈之下,电视台负责人匆忙赶来,花费重金包下了录制谢小蝉唱歌并将之直播的权限。村书记头一次赚得如此盆满钵满,他精心给谢小蝉准备饭菜,并小心地剔除一切对嗓子有害的东西。而谢小蝉毫不在意树下的人所作的一切,她仍然只是躺在树枝上唱歌。她的歌声没有发生任何的变化,在无人聆听她唱歌时她如此唱,在树下聚集了无数人、千里之外无数人在听她唱歌时,她也是那么唱。

策划对负责人的这种行为感到不满,在她看来,没有优胜劣汰,没有搞笑台词,没有让人落泪的故事,单纯的一个在树上唱歌的噱头,怎么能维持住那么高的人气呢?但现实确实印证了负责人的眼光,这个新推出的电视节目取得了巨大的成功,而镜头不过是在拍摄一大片浓密的树叶,以及依稀露出的人影而已。

通过她的声音,那混杂着无数聒噪的蝉鸣的歌声,她一下子变得全国出名。又过了一两天,这件事传到了国外,整个世界都沦陷在她的声音里了。

全球的各种音乐家、演唱家、国内外著名的歌手都来到了这个小村庄,然后在那里交流这种歌声。

“Is cineál nua amháin chomhchuibhithe é seo.”一个外国人在那里嘀咕。

“我建议将其命名为‘流蝉’风格。”

这些专业人士同其他人一样,站在那里,沉浸在谢小蝉的歌声里,他们的每次讨论都躲得远远的,生怕打扰了声音的来源。

一位奥地利的中年男子曾经走到那棵树下,在她吃饭的时候,借助翻译请求谢小蝉去维也纳金色大厅独唱。

谢小蝉只是问道:“金色大厅,建立在一棵树上吗?”

“当然不是。”翻译流畅地回答。

“哦,那抱歉。除了这棵树,我哪里也不去。”

她直接用手将饭盒里的水饺和韭菜盒子一一放进嘴里,她已经可以说话了,虽然中文说的很不标准。她仍然不愿意过多地发言,除了保持沉默,就是在唱歌。

如果你在现场,你可以发现她的头发渐渐长了出来,有点像寸短,她的头发有点发黄,不知道是不是太阳的原因。

她在树上排泄,然后经由工作人员搬运。我觉得她本可以自由,却又被束缚在这棵树上。但她却从未流露过这种情绪,她似乎在那个夜晚,在她爬上树的时候,就做好了所有的准备。

来到村子的人越来越多,尖叶树已经挤不下了。县里设置了警戒线和安检关卡,在尖叶树周围安放了大量的摄像头和声音采集装置,将人群驱散到广场上,仅仅让重要的人进出尖叶树领地。

不到一个月之后,她死了。

然后这个荒诞的故事就此结束。音乐家们还在讨论如何发出她的这种歌声,生物学家还在研究她的基因究竟在哪些方面具有独特性,我还把她留在我的梦里。而对于大多数人,他们又去追逐下一个热点,某个国家的电鳗,北海的某条鲲,哪个明星又发明了一种新颖的多人运动方式,某个国家发生了暴乱,烧毁了一座座城市,哪里又出现了腐败和特权。在他们的记忆中,似乎这样一个人,这样一件轰动世界的事,从来从来,就没有发生过。村里的人也一样,不出两三年,甚至不过两三个月,大家便都将这件事忘记了。人们又开始为了生计奔波,生孩子,养孩子,操心房子,操心结婚,操心身体,最后做不动了,埋在黄土里死去。

没过多少年,村落整体直接被搬走,安置在郊区新盖起的、没人接盘的楼市里。没有多少人觉得遗憾,挣钱仍然是一个难题。再后来,那棵树也被砍掉了。我不知道谢小蝉在上面留下了什么。石溪的水逐渐干涸。 

\section{第八章}
\label{sec:org45c4814}

我的父亲今年六十多岁。他的身体很好,只不过血压有点高。

和我居住在一起几年,他总是不满意。他还是想回到他一辈子生活的地方去住。甚至,他还在幻想经营一家餐馆。从附近村的牧羊人手中买下几只羊,自己动手剥皮,切肉,将骨头和肉块放进大锅里煮熟。他一生就会这一个技能,煮上一大锅羊肉汤。

他还会打烧饼,将面做成矩形,放入炭火中烤熟,就着羊肉的香味品尝麦香。

他在城市里呆着的几年,总是在念叨他的小餐馆。从兴旺到无人问津,当生意惨淡了之后,唯一的转机,便是二十岁的夏天,谢小蝉唱歌的季节。

那时候,各处的人都来到村子里参观,也大都去他那里吃饭。他从未接纳过那么多人,每天都去联系卖家买羊,然后在店门口杀掉,在露天场地上支起一个大铁锅煮肉。全世界的人都过来品尝他的成果,嚼咽他的烧饼,冒着汗大口大口喝着他做的汤。

每晚他都在笑着抱怨:“啊呀,人怎么这么多!难道是要累死我不成吗?”然后第二天兴奋地早早起来,又去联系牧羊人买羊。

我也帮我父亲干活,多是称羊肉,然后端上去盘子,或者收钱这类简单的活。有一天,陈昊竟然也来了我们家的餐馆。

“两斤羊肉,一斤烧饼。一斤一碗。”我原以为另一碗是给他女朋友乘的,可她为什么没和他一起来呢?看眼神,他似乎很重视那个人。他等得有些着急,一直在看手机的消息。

我那时候或许确实对他不爽,羊肉虽然没有缺斤少两,但掺杂内脏的量却比寻常多了一些。

半个小时后,他等的人才过来。西装,皮鞋,领带,公文包,大肚便便。看脸色那人也就不到四十岁,但头发已经完全灰白了。

“教授您好,欢迎来我家乡!”陈昊本身就是能说会道,

那个教授也很客气:“你刚入大学,就对科研这么感兴趣,做实验也进步很大,喏,我现在到你家乡考察来了!”

然后他们两个人就开始闲谈起来,话题很快就转到了谢小蝉上。

“你认为,她的这种性状的变化情况:我是说,从毛发和发声两个角度共同来看,是什么原因引起的?”

“我觉得不明晰。”陈昊说道,“如果非要给出一种假设的话,我觉得更像是先天的问题。”

“先天的问题?”

“对。”他一边吃着烤排,一遍说道:“我记得她是被遗弃的孩子。很可能她的父母在生下她之后,便从体检报告中得知了什么。我怀疑是基因的问题。”

“有这个可能。”教授点点头,他似乎对食物没有兴趣,可是羊肉汤真的很好吃。“你还有什么想法?”

“另一个想法当然更奇特一点。”陈昊说道。“目前的生物进化理论仍然是在达尔文自然选择这一核心的基础上构建的,而真实的问题是,用进废退或许真的可以存在。以前人们以为,自身的锻炼难以改变自身的基因,因而总是无法传递到下一代。但是,单从我们这个时代看,近视的人生下的孩子生来就已经近视了,或者是眼睛出现了其他的一些异常。我们有必要针对这个特殊的形象进行一番实验,如果能够发掘出任何新的东西,都将是Nature级别的工作。”

教授点了点头。“除此之外,在这个个体上更神奇的一点是,发育的变态性。人作为哺乳动物几乎不存在变态发育的可能,如果她的这次由哑变成会唱歌,身体重新长起毛发不是第二性特征发育的附属品的话,那么这也是非常重大的一个发现。”

“我可以确定,她不是青春期。”陈昊直接答道。“她的青春期和大多数人一样,早就过去了。”

“你怎么知道?难道她的性特征很明显?”

“我……”陈昊放下筷子,呛了一口汤,“我曾经和她是恋人关系,她堕过孩子。”

“快去结账!二维码呢?”我似乎没有听见父亲骂我的话,我也可能听见了。我便去给客人算钱,结账。

饭点一过,店里立刻冷清了一半。我不想在呆在这里,便又失魂落魄地在村里散步,最后,我又走到了尖叶树那里。

她还是谢小蝉吗?半个多月过去了。她的头发疯狂地生长着,如同豆苗地里的荒草。她还是躺在最大的那根树枝上,任由头发往下耷拉着。她的头发变得出奇的有光泽,乌黑浓亮,再也不是之前稍稍发黄的样子了。其中,有些成liu的地方就在那里把头发稍稍聚结,仿佛是一株吊兰。她还是她吗?冷静,对一切世事都无所谓,然后在那里,哼着没有人不会沦陷的歌。

那位教授似乎获得了谢小蝉的许可。他们开始采集她的头发,粪便,尿液,然后又把非植入式的传感器放在她的皮肤上,头发边,试图采集体内的电波(陈昊是这么说的)。他们动用了各种医疗仪器对她进行检测。她对这一切似乎满不在乎,仿佛那些仪器的机械臂都是一些新式的树枝而已。她还是躺在那里,唱她自己的歌。

再然后,教授和陈昊也开始忙碌起来,他们天天在那里讨论着这种现象究竟是何种起因。

接着,另一队科学家也到来了。他们摆起辐射光台,似乎想通过分析她脖子处的发音部位来给出能产生这种优美声音的科学解释。他们并没有找到多少靠谱的解释,唯一探究清楚的是,为什么她可以唱出这么好听的声音。那些人说,人的声带膜都是受到振动而发出声音,而她的声带膜,因为接受过多年的力量训练而变得出奇地有力,鸣肌甚至每秒能伸缩约1万次,而喉咙和声带膜之间的微小空腔,又能起共鸣的作用,所以她的声音如此地悠长美妙,而她又从不知疲倦。就这样,这些科学家不虚此行,成功发了一篇Nature。

不过,我对这个解释一头雾水。他的喉咙接受过力量训练吗?难道在她无法说话的时候,她一直在做这种尝试?

陈昊那边的进展好像没有那么顺利,他竟然和教授吵起来了。

那时候,在专业的伙食队和各类专业的科学家到来之后,负责监测她身体健康的人也来到了这里。他们对她的身体指标进行了另外的分析,得出的结论是:“谢小蝉必须每周唱歌的时间不超过10小时,否则,她的声带将发生不可逆转的损坏,最终倒向悲剧。”这可真是个及时的提示。电视节目的负责人,村里的书记们都开始央求她,告诉她一直唱歌,一直唱下去的可怕。他们一边在为未来的歌星的健康着想,一边又为丢掉这棵摇钱树后悔。

紧接着,医疗队又发现她的心脏承受着和她的声带一样的负担。这代表如果她不停止一直唱歌的话,她甚至就要永远地在第二十个年头死去。

“我并不在意出什么事。”她平静地说道,“我只不过是想唱歌。可能哪天,我失去了这个乐趣,就不会唱了。”

话虽如此,每一天,她还是在那里哼着世界上最好听的歌。医疗队开始进行估计,他们给出结论:“谢小蝉如果再维持这个强度下去,她甚至撑不过这个夏天。”

谢小蝉倒对此毫不在意,她会不会早就想到这些了呢?在她决定上树的时候,她或许就想到了很多很多事情吧。

不过,陈昊,和那位教授,似乎成功达成了一个协议。他们走到那棵尖叶树下,向她问起话来。

“你好,谢女士。”他非常之客气,“按照医生们的看法,您可能很快就要去世了。”

“是的,这位先生。”

‘“那么女士,如果您这样美丽的声音无法在世界上流传,那岂不是一件非常可惜的事吗?”教授说道,“我们在现场的各位,十分幸运,可以听到这样的天籁之音,听到您现场独具生面的演奏。但是,如果您离开了这个世界,后世的所有人只能在录音里回味您的歌声,这难道不太令人遗憾了吗?”

“先生有什么解决方案吗?”

“当然。”教授说道,“您能唱出来这么美妙的曲子,除了您具有独特的演唱技巧之外,您的生理特征,也是极其独特的。后世的人或许可以学习您唱歌的技巧,但是这种生理特征却是无法具备的,而传承这种生理特征的方式,只有遗传。”

“遗传。”她回味了这两个字,“就是,生孩子吗?”

“是的!”教授答道。“我可以确信,您的儿子,或者女儿,肯定是世界上最伟大的音乐家,它将成为音乐界最大的宠儿。”

“是啊。”谢小蝉说道,“它或许十九年都不能说话,当一个哑巴。十九年没有毛发,被别人嘲笑。然后,它的一生只能当一个音乐家——生来如此的音乐家。就像皇帝的儿子必须被束缚在皇宫里做太子,奴隶的儿子还是奴隶一样。”

“这可是无数人都在想的好事!”教授说道,他的声音激动了。“难道你在世界上就没有一个愿意爱的人了吗?我相信,任何人,都会愿意爱你并且和你组建家庭的!任何人!”他说这句话的时候看了看陈昊,后者点了点头。

谢小蝉坐了起来。她坐在树杈上,散着头发,因为炎热而光着一些身子。他把树下站立着的人都扫视了一眼,她的眼神十分平静,不仅仅是陈昊,当我和她的眼睛对视时我也觉得被电击了一下。她看了一遍,然后又躺着了。

“我有一个深爱着的人,”她说道,“他在这个春天就已经死了。”

到现在我都不知道,她说的这个深爱的人,是指她爷爷,还是过去的陈昊。我多么希望,是我啊。

总之,一日接着一日,盛夏已经过去了。蝉声渐渐地稀少起来,但仍然十分聒噪。她的嗓子似乎也稍稍有一点点的喑哑。不过这种沙哑反而让她的声音更美丽,更加让人陶醉。人们已经无可奈何地接受了她会死去的结局,全国最专业的声音采集团队也过来了。他们隔离了方圆很大的一片距离,试图最好地展现这将流传后世的声音。

采集工作进展得比较顺利,各种麦克风和回音墙都建立了起来。唯一令人头疼的是,任何的一段音频里都混有蝉鸣,尖锐、嘈杂、令人不适的蝉鸣。

他们一边进行声音的采集工作,一边抽调了数字信号处理领域的专家来对这些声音进行清洗。据他们在吃羊肉汤和烧饼时的聊天,无论他们怎么进行频率分解和滤波,怎么进行插值和卷积降噪,都无法根除那些蝉鸣。那些蝉鸣甚至霸占了她歌声中的各个频率,时而梯度为零时而梯度无穷大的蝉鸣声一经清洗,反而越加让人觉得不适。按照那些专家的话来说,会产生一种“指甲划黑板的声音”。

那些人最后接受了这个现实,哭丧着脸完成了采集。

夏天过去了,我也已经开学,又离开了我的家乡。听父母说,即使秋天来临,谢小蝉也没有死。但她的嗓子已经沙哑得不像话了,乌黑的头发也开始渐渐发白。她不吃不喝,最后死在了树上。

\section{第九章}
\label{sec:org7a78b32}

我和我妻子相识是在工作以后。她是一个十分安静的人。我不知道我爱上她是不是因为这种安静,我甚至不确定我对她的爱情是来自于本心,还是来自于时间的洪流。她一般扎着一个辫子,有段时间也理着短发,有段时间也披着头。和我一样,她也戴着矫正近视的眼镜,鼻梁稍稍有些塌陷,两只眼睛不算特别大。生下孩子之后,她胖了一些。肚子上布满了妊娠纹,肚脐眼陷得很深。我不知道我是不是爱她,她已经融入我的血肉,成了我的亲人。

我和她经常吵架,赌气,但每次从夜晚中因为什么事惊醒,我都愿意抱着她睡着。

我从未同她讲过谢小蝉的事,不知道怎么开口。我甚至很少提及我童年的往事,小学发生过的,初中发生过的,高中发生过的,大学发生过的往事。今年冬天,我又回到阔别多年的家乡,和衰老的父母一起过年。后来因为疫情,迟迟未返。

我和我妻子经常漫步在早就搬走、无人居住村落里,我们彼此都不说话,仅仅是牵着手,安静地往前走。水泥路早就开裂,里面长出野草和野花。盛夏的蝉声依然密集,我早就找不到尖叶树了,甚至树桩都已经发黑,腐烂。几年前那里曾经有一个光着头的姑娘笑着和我打招呼,她曾经爬上村里最古老的那棵树,在上面唱着没有人不会被感动的歌。

那天晚上,台风还未来临。我在槐树下小憩,一边用手打着蚊子,一边构思新创作的小说。她走进我家大门,在狗吠声中递给我一本书。

我接过那本书,想问她生活是否顺利。我没办法问出来,我和她已经不熟了,况且她的爷爷去世,她知道我知道。

她又开始比划起来,似乎是问我外面的世界如何,有没有什么令人激动的事。

“还不错!”我只是应付着说道,“当然,没有这本书中故事那么有趣。”

她扑哧一声笑了出来,嘴巴还是半张着。我没在意,那时候她就已经可以说话了。

她从口袋里拿出钢笔,然后在书的扉页上写道:“今天听说有台风,你要和我一起去尖叶树上玩吗?”

“这么大的风!”我觉得奇怪。“还有雨呢。我不去,你也不要去。”

她平静地笑着,又在上一行字的下面写道:“我偏要去!”然后扣上钢笔,把书丢给我,跑出门去了。

“这里没有鱼!什么鱼也没有。”他在那里嘀咕。“爸爸,你不是说爷爷住的地方都是鱼吗?这里明明什么也没有……”

他还顺着那摊水往前走,一边走一边说道:“果然,你在骗我,你在骗我。你觉得我是小孩子,所以骗起来就很有乐趣,所以骗起来就很容易?这里明明没有鱼,更没有虾,也没有螃蟹,这里什么也没有……”



结束。。。。。


\section{后记}
\label{sec:orgeed4a2e}

本小说大约是某次中午散步时,听着蝉声构思的故事。五月初五的一个傍晚,我沿着县城边界新修建的外环公路骑车游玩。我游历过一个村子,村里的人整体搬走了,只剩下几户无论如何都不愿意离开的老人。里面有一棵非常老的核桃树,旁边有一个歇息的老奶奶。老奶奶非常健谈,把她的所有情况以及村子里的所有情况全给我说了一个遍。她说她七十二岁,在她出嫁到这个村子时,这棵核桃树就存在了。核桃树上有一个明显的大洞,里面满是黑色的颗粒,不知道是什么东西。我又去旁边早已干涸的河中看了半天。里面还蓄有一些水,没有水草和青苔,水和石头都十分干净。一个人被四周的小树包围,望着河水,就如同柳宗元笔下的小石潭。

旁边树上的的蝉鸣特别响,每个走近的人都会感到极度的不适。我在那里被这些蝉鸣包围,便想起了之前构思的小说,下定决心写下去。

回家的路上看见一对青年夫妇在那里散步,便借助那个男人的口吻写了这个故事。那个村村口有卖羊肉汤的野店,我舍不得花钱吃,好烦啊。而文中主人公的那些童年的朋友,我承认在洗澡、偷东西这些事情上有一定的影子,但是我的玩伴和他们完全不同。那些和我光着屁股玩大的男孩子,都结婚的结婚,成家的成家了。他们都在临沂生活,所以关系很近。端午还去海边玩。他们估计已经不太关心我了,什么事也不叫我,我只能一个人拜访无人的村庄。那些女孩子,我没有对她们产生任何的情愫,可能是兔子不吃窝边草的习俗吧。我邻居专升本结束,似乎是嫁给了一个新疆人。小学的同桌结婚去了城北,我还参加了她的婚礼。希望他们都过得不错。

这个故事很显然是不会存在的。造化不会这么无情,产生一个谢小蝉这样可怜的人吧。我本来想把它写成科幻小说,可是这个东西一点也不科幻。我又想把它写成《彩虹症》这种风格的童话故事,但是这里面的悲剧有点多,也没法变成童话。然后我又尝试将其转变为《小公主》类似的乡土风格,可是,乡土风格,我哪里能理解乡土风格。我仅仅是把上面这三种东西杂糅在一块,搞出了一种四不像罢了。


\textit{<2020-06-28 周日>}
于 费县老家
\end{document}