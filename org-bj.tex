% Created 2020-06-16 周二 17:48
% Intended LaTeX compiler: pdflatex
\documentclass[10pt,a4paper]{article}
\usepackage{graphicx}
\usepackage{xcolor}
\usepackage{xeCJK}
\usepackage{lmodern}
\usepackage{verbatim}
\usepackage{fixltx2e}
\usepackage{longtable}
\usepackage{float}
\usepackage{tikz}
\usepackage{wrapfig}
\usepackage{soul}
\usepackage{textcomp}
\usepackage{listings}
\usepackage{geometry}
\usepackage{algorithm}
\usepackage{algorithmic}
\usepackage{marvosym}
\usepackage{wasysym}
\usepackage{latexsym}
\usepackage{natbib}
\usepackage{fancyhdr}
\usepackage[xetex,colorlinks=true,CJKbookmarks=true,
linkcolor=blue,
urlcolor=blue,
menucolor=blue]{hyperref}
\usepackage{fontspec,xunicode,xltxtra}
\setmainfont[BoldFont=Adobe Heiti Std]{Adobe Song Std}  
\setsansfont[BoldFont=Adobe Heiti Std]{AR PL UKai CN}  
\setmonofont{Bitstream Vera Sans Mono}  
\newcommand\fontnamemono{AR PL UKai CN}%等宽字体
\newfontinstance\MONO{\fontnamemono}
\newcommand{\mono}[1]{{\MONO #1}}
\setCJKmainfont[Scale=0.9]{Adobe Heiti Std}%中文字体
\setCJKmonofont[Scale=0.9]{Adobe Heiti Std}
\hypersetup{unicode=true}
\geometry{a4paper, textwidth=6.5in, textheight=10in,
marginparsep=7pt, marginparwidth=.6in}
\definecolor{foreground}{RGB}{220,220,204}%浅灰
\definecolor{background}{RGB}{62,62,62}%浅黑
\definecolor{preprocess}{RGB}{250,187,249}%浅紫
\definecolor{var}{RGB}{239,224,174}%浅肉色
\definecolor{string}{RGB}{154,150,230}%浅紫色
\definecolor{type}{RGB}{225,225,116}%浅黄
\definecolor{function}{RGB}{140,206,211}%浅天蓝
\definecolor{keyword}{RGB}{239,224,174}%浅肉色
\definecolor{comment}{RGB}{180,98,4}%深褐色
\definecolor{doc}{RGB}{175,215,175}%浅铅绿
\definecolor{comdil}{RGB}{111,128,111}%深灰
\definecolor{constant}{RGB}{220,162,170}%粉红
\definecolor{buildin}{RGB}{127,159,127}%深铅绿
\punctstyle{kaiming}
\title{}
\fancyfoot[C]{\bfseries\thepage}
\chead{\MakeUppercase\sectionmark}
\pagestyle{fancy}
\tolerance=1000
\date{\today}
\title{}
\hypersetup{
 pdfauthor={},
 pdftitle={},
 pdfkeywords={},
 pdfsubject={},
 pdfcreator={Emacs 26.3 (Org mode 9.1.9)}, 
 pdflang={English}}
\begin{document}

\tableofcontents

 这篇笔记是我用org写的,在学习org的过程中进行的一个简单的记录。
这篇笔记里所有的内容都来自于\href{https://orgmode.org/guide/}{orgmode compact guide}.
\section{{\bfseries\sffamily DONE} introduction: a welcome of org mode}
\label{sec:orgf32fdbd}
\subsection{激活}
\label{sec:org3ee883d}
当你初次使用一个emacs,且你并没有什么配置的时候,如何从零开始配置org呢?首先,尝试将以下代码复制到init.el文件里,当然,也可以是合理的其他位置.这样做的目的是为了激活快捷键.
```
(global-set-key (kbd "C-c l") 'org-store-link)
(global-set-key (kbd "C-c a") 'org-agenda)
(global-set-key (kbd "C-c c") 'org-capture)
```
\section{{\bfseries\sffamily DONE} document structure 文档结构}
\label{sec:org89271e5}
文档结构被认为是文档的骨架,也就是一个"书"状的层次结构.
\subsection{headlines}
\label{sec:orge7670f7}
就是一级标题,二级标题等等.一般可以通过以下方式进行表达:


\begin{verbatim}

* 一级标题
** 二级标题
*** 三级标题

\end{verbatim}


此外,可以通过"M-<ENTER>"键一键形成一个同等的一级标题.可以使用TAB将这个一级标题转换为一个二级标题.
一般,当打开一个org文档时,这个org文档仅仅会展露出一个骨架.此时可以通过TAB将这个骨架进行展开.
\subsection{在可见度上的遮盖与打开 visibility cycling}
\label{sec:orgee25423}
也就是在可见度之间的一种循环.前面有所介绍.
\begin{enumerate}
\item 最常用的方法是使用TAB.如:
\end{enumerate}
,-> folded FOLDED -> children CHILDREN -> subtree SUBTREE --.
'-----------------------------------------------------------'
\begin{enumerate}
\item 使用S-TAB在以下场景下实现循环.
\end{enumerate}
,-> OVERVIEW -> CONTENTS -> SHOW ALL --.
'--------------------------------------'
\begin{enumerate}
\item 使用C-u C-u C-u TAB ,实现show all 的功能.
\item 自定义一个org文档起始时刻应该具有的结构.
\end{enumerate}
一般而言,可以在org文档的开头这么写:

\begin{verbatim}

#+STARTUP:content

\end{verbatim}
还可以设置变量比如:overview,content,showall等.
\subsection{在headline之间的跳动}
\label{sec:org41fdbe3}
有的时候,是想直接在headline之间进行跳动的.这些过程通常可以经由以下快捷键进行展示.
值得注意的是,这些快捷键显然是C-c加上了一些独特的后缀.
C-c C-n Next heading.从当前文本跳跃到上一个headline处,或从当前的headline跳跃到上一个headline处,而不论上一个headline是否与此处的headline同级别.你可以通过这个按钮跳跃到与光标相比最近的上一个headline处.
C-c C-p Previous heading.类上
C-c C-f Next heading same level 只会在同一level的headline之间跳转,并且归于他们的上级那里,出不去.
C-c C-b provious heading same level类上
C-c C-u backward to higher level headings.?
\subsection{结构编辑 structure editing}
\label{sec:org3b00968}
结构编辑主要存在以下快捷键.
\begin{enumerate}
\item M-RET 添加一个同级别的headline
\item M-S-RET 添加一个同级别的todo headline
\item M-LEFT M-RIGHT 将当前headline升级或者降级
\item M-UP M-DOWN 将当前headline同其包括的所有内容上移或者下移
\item C-c C-W 将本healine的所有内容归属到另一个一级标题之下
\item C-x n s C-x n w 在buffer层面进行移动
\end{enumerate}
\subsection{sparse trees}
\label{sec:org9159014}
   sparse tree 是一种有侧重地进行"目标选择"的工具.(不太确定,我目前这样理解这一功能)
针对这种工具,基本的使用方法有:
\begin{enumerate}
\item C-c / 这可以打开一个sparse tree 按钮
\item C-c / r 关键字搜素.比如,在本文中,搜素和展示有关headline的内容.
\end{enumerate}

\subsection{plain list 简单的列表}
\label{sec:orgdb78b68}
简单的列表可以通过以下标记符号进行快速地创建.
使用"-" "+" "*" 进行无序号列表的创建,使用"1." "1 "进行有序号列表的创建.使用"::"进行解释.
下面是一个例子.值得注意的是,这里"::"充当的作用,与latex中\label极为相似.二者都是在给出一个方便于引用的对象.关于如何应用之,可以看\ref{sec:org296a1d0}.

\begin{verbatim}

* Lord of the Rings
  My favorite scenes are (in this order)
  1. The attack of the Rohirrim
  2. Eowyn's fight with the witch king
     + this was already my favorite scene in the book
     + I really like Miranda Otto.
  Important actors in this film are:
  - Elijah Wood :: He plays Frodo
  - Sean Astin :: He plays Sam, Frodo's friend.

\end{verbatim}
\section{{\bfseries\sffamily DONE} table 表格的使用}
\label{sec:org6479a10}
表格的使用主要通过"|"符号实现
一般一个表格是需要通过这样子完成的
\begin{center}
\begin{tabular}{lrr}
Name & Phone & Age\\
\hline
Peter & 1234 & 17\\
Anna & 4321 & 25\\
 &  & \\
 &  & \\
\end{tabular}
\end{center}

\begin{verbatim}
| Name  | Phone | Age |
|-------+-------+-----|
| Peter |  1234 |  17 |
| Anna  |  4321 |  25 |
|       |       |     |
|       |       |     |

\end{verbatim}

但是,显然,可以看出,这样的表格无法进行高效的输入,因为中间那行长长的横线很烦人.解决方案通常是:当你输入了"|-"之后,直接使用TAB进行自动补充.除此之外,你也可以通过TAB形成一个新的填空.
\subsection{使用C-c | 形成一个新的表格}
\label{sec:org14200f8}
如题所述,虽然不怎么常用.
\subsection{cell基本变换}
\label{sec:orga08ba6c}
\begin{itemize}
\item C-c C-c 在不移动点的前提下重新对齐表格
\item TAB 横向,移动到下一个
\item S-TAB 横向,前一个
\item RET 下一行
\item S-方向键 让当前的cell和周围的某个cell进行交换
\end{itemize}
\subsection{{\bfseries\sffamily DONE} 行与列的变化}
\label{sec:org2845e66}
行与列的变换都是基于"M"进行的.
\begin{enumerate}
\item M-LEFT M-RIGHT 将当前的列左移或者右移
\item M-UP M-DOWN 将当前行上移或者下移
\item M-S-LEFT 删除当前列
\item M-S-RIGHT 插入新列
\item M-S-UP 删除当前行
\item M-S-DOWN 插入新行
\item C-c -, C-c RET 分别表示插入一条horizontal line,在下面,或者上面
\item C-c \^{} 列排序
\end{enumerate}

\section{{\bfseries\sffamily DONE} hyperlinks 超链接}
\label{sec:org8c44e57}
超链接,不用多数,一般遵循[ [link] [description] ] .对其进行编辑,可以通过C-c C-l进行.

\subsection{内部链接}
\label{sec:org88a0308}
内部链接这里作者并没有给出详细的阐述.笔者尝试了以下,对于特殊的一些格式似乎都是可以识别的.

\subsection{外部链接}
\label{sec:org296a1d0}
首先,罗列一些典型的外部链接:

\begin{verbatim}

http://www.astro.uva.nl/=dominik’	on the web
file:/home/dominik/images/jupiter.jpg’	file, absolute path
/home/dominik/images/jupiter.jpg’	same as above
file:papers/last.pdf’	file, relative path
./papers/last.pdf’	same as above
file:projects.org’	another Org file
docview:papers/last.pdf::NNN’	open in DocView mode at page NNN
id:B7423F4D-2E8A-471B-8810-C40F074717E9’	link to heading by ID
news:comp.emacs’	Usenet link
mailto:adent@galaxy.net’	mail link
mhe:folder#id’	MH-E message link
rmail:folder#id’	Rmail message link
gnus:group#id’	Gnus article link
bbdb:R.*Stallman’	BBDB link (with regexp)
irc:/irc.com/#emacs/bob’	IRC link
info:org#Hyperlinks ’	Info node link

除此之外,还有一些特殊情况,这些特殊情况包括:
file:~/code/main.c::255’	Find line 255
file:~/xx.org::My Target’	Find ‘<<My Target>>’
[[file:~/xx.org::#my-custom-id]]’	Find entry with a custom ID

\end{verbatim}

\subsection{handling links, 处理链接}
\label{sec:org5f7501d}
\begin{enumerate}
\item C-c C-l 插入一个链接.当该处存在链接时,其意义是修改一个链接.
\item C-c C-o 打开一个链接.
\end{enumerate}
\section{{\bfseries\sffamily DONE} todo iteems  待办项目}
\label{sec:orga069fbb}
\subsection{有关todo的基本操作}
\label{sec:orgf902297}
当一个items的前面包含todo的时候,它就变成了一个todo 的item.
一般而言,todo的基本命令如下:
\begin{enumerate}
\item C-c C-t 打开todo选项.
\item S-左右 cycling todo的状态吧.
\item C-c / t  在sparse tree里看todo.有关于sparse tree的信息参见sparse tree.
\item M-x org-agenda t 展现出全局的todo
\item S-M-RET 输入一个新的todo.
\end{enumerate}

\subsection{{\bfseries\sffamily DONE} muti-state workflow  多态工作流}
\label{sec:org5b57342}
muti-state指的就是"并非所有的待办都是todo->done"循环的产物.比如debug的过程,可能是下面的形式.


\begin{verbatim}
(setq org-todo-keywords
      '((sequence "TODO(t)" "|" "DONE(d)")
	(sequence "REPORT(r)" "BUG(b)" "KNOWNCAUSE(k)" "|" "FIXED(f)")))

\end{verbatim}


这时,简简单单使用todo这一套就不太管用了.我觉得这里的东西没什么太多的实际用途.
\subsection{Progress Logging 进展记录}
\label{sec:org390dde6}
进展记录,最简单的使用方法是通过引入一个前缀"C-u",来加入一个时间戳.也就是通过"C-u C-c C-t"来改变todo项目的状态.
emacs里面有专门的时间记录,详细可参阅\href{https://orgmode.org/guide/Clocking-Work-Time.html\#Clocking-Work-Time}{此处}.
\subsubsection{{\bfseries\sffamily TODO} 阅读clocking working time}
\label{sec:org8f71380}
\subsubsection{closing items 关闭项目}
\label{sec:org0e02f74}
通过引入
(setq org-log-done 'time)
使得每次有一个item被标记为done之后,都会插入一个时间戳.
同样地,也可以通过引入
(setq org-log-done 'note)
在结束项目的地方插入一行注释.
\subsubsection{tracking todo state changes}
\label{sec:orgd19a0d5}
没兴趣做.略.
\subsection{Priorities 优先级}
\label{sec:org64f4be3}
就是对todo设置优先级的问题.一般优先级会用ABC进行表达.
\begin{enumerate}
\item "C-c ,",设置优先级,可以输入ABC.通过空格键进行移除.
\item S-上下 改变优先级.
\end{enumerate}
\subsection{break tasks down into subtasks 将任务分解为子任务}
\label{sec:orgadda3f8}
在父标题下使用[/]或者[\%],之后,在子标题里设置todo的状态,就可以了.
\subsection{checkboxes 复选框}
\label{sec:orgb8f037e}
在使用plain list的时候,可能会用到这个功能来进行进度管理.
比如下面的例子:

\begin{verbatim}

* TODO Organize party [1/2]
  - [ ] call people [0/2]
    - [ ] Peter
    - [ ] Sarah
  - [X] order food
使用C-c C-c来进行checkboxes状态的切换.

\end{verbatim}

\section{{\bfseries\sffamily DONE} Tags 标签}
\label{sec:org10ea6f6}
标签是用来进行交叉引用的一类东西,标签类似于完成latex里label的功能.标签一般被放在headline的后面,前与后都用":"作为连接.下面是一个简单的例子.

\begin{verbatim}
* Meeting with the French group      :work:
** Summary by Frank                  :boss:notes:
*** TODO Prepare slides for him      :action:
\end{verbatim}

\subsection{tag inheritance 标签层级}
\label{sec:org6db5e86}
以上面的例子为示,标签的层级具有一定的关联性.比如最后的headline,它包含着所有的标签,也就是,他继承了他的父标题以及祖父标题的标签.

当然,也可以在文章中定义标签,这种定义方法为:

\begin{verbatim}
#+FILETAGS: :Peter:Boss:Secret:
\end{verbatim}

\subsection{设置标签\hfill{}\textsc{test}}
\label{sec:org16722eb}
\begin{enumerate}
\item M-TAB 无法使用,与系统的页面转换重合
\item C-c C-q 为当前的headline插入一个tag
\item C-c C-c 当光标在headline时,同2
\end{enumerate}

除了前面那种一个个插入标签的方法之外,org支持插入一个标签列表,其基本语法为:

\begin{verbatim}
#+TAGS: @work @home @tennisclub
#+TAGS: laptop car pc sailboat

\end{verbatim}

除此之外,emacs支持快速标签选择,也就是一个按键输入一个标签,这需要在配置文件中写入:

\begin{verbatim}
(setq org-tag-alist '(("@work" . ?w)
		      ("@home" . ?h)
		      ("@laptop" . ?l)))
\end{verbatim}

\subsection{标签组}
\label{sec:org4284b6f}
标签组是很多个标签组成的集合.他的用途是:当进行标签的搜索时,如果输入了标签组的名字,那么就可以返回匹配标签组内所有标签headlines
标签组的定义方法如下.

\begin{verbatim}
#+TAGS: [GTD : Control Persp]
#+TAGS: {Context : @home @work}
\end{verbatim}

\subsection{标签的搜索}
\label{sec:org6951b70}
\begin{enumerate}
\item C-c / m or C-c $\backslash$  生成一个sparse tree,
\item M-x org-agenda m  通过agenda file 生成一个全局的标签匹配列表
\item M-x org-agenda M  在2的基础上,仅仅显示带有TODO标签的那些.
\end{enumerate}

值得注意的是,这些标签均支持布尔运算.比如使用"a+b-c"代表包含a标签并包含b标签且不包含c标签的所有匹配项.使用"x|y"代表包含x标签或包含y标签的匹配项.


\section{{\bfseries\sffamily DONE} Properties}
\label{sec:org288e0ac}
properties类似于一种“面向对象”的使用方式,也就是定义了一个实体,下面有诸多变量,并依据这些变量具有某些特定的数值来描述其属性。
鉴于很无聊,就将其略去。
\section{{\bfseries\sffamily DONE} dates and times}
\label{sec:org8b32537}
\subsection{timestamps 时间戳}
\label{sec:org585b12c}
此处存在各种各样格式的时间戳,然而,对我而言,这并非需要关心或者讨论的重点,因而对其仅进行简要介绍.
\subsubsection{C-c . 插入时间戳}
\label{sec:orgf865436}
这个命令用来插入一个时间戳,(如果有时间戳了,那么就是修改这个时间戳).连续使用两次这个指令可以形成一个时间戳的范围,在这个范围之内可以完成一些或许更加一般的事.
\textit{<2020-06-07 周日>--<2020-06-16 周二>}
\subsubsection{C-c ! 插入非活动类型时间戳}
\label{sec:orga22bc51}
这个命令插入的时间戳不会被调用在agenda里面.
\subsubsection{S-方向键}
\label{sec:orgb3cd156}
控制上下左右,似乎有一些独特的细节,不过我不关心.
\subsection{deadline and scheduling 截止日期与时间表}
\label{sec:orgf0ee33b}
\subsubsection{C-c C-d}
\label{sec:org742dffe}
这样就直接输入了一个deadline.

\subsubsection{C-c C-s}
\label{sec:orgd2d7739}
schedule是一种描述一种东西什么时间开始的日期.
[测试了,无法使用.]

\subsection{clocking work time 记录在特定项目上消耗的时间}
\label{sec:org9a22eb8}
如题所示,这一章来看一看如何记录消耗在特定项目上的时间.
\begin{enumerate}
\item C-c C-x C-i 打开一个clock(clock in)
\item C-c C-x C-o 关闭一个clock(clock out)
\item C-c C-x C-e 升级当前时钟的估计工作量
\item C-c C-x C-q 退出当前时钟,如果不小心打开了一个时钟,可以用这个选项
\item C-c C-x C-j jump,跳转到任务中当前计时的标题
\end{enumerate}
\section{{\bfseries\sffamily INPROGRESS} capture, refile, archive}
\label{sec:orga8d9bb2}
\subsection{capture}
\label{sec:org1571fb6}
capture(名词,捕捉): capture是指在知识系统中快速捕捉新的主意与任务(task)的一种方式。并且,这种捕捉还可以关联与其相关的一些材料。这一整套的流程被称作capture。
\subsubsection{setting up capture 设置capture}
\label{sec:orge3f6ed4}
可以通过下面命令设置默认的笔记路径。

\begin{verbatim}
(setq org-default-notes-file (concat org-directory "/notes.org"))
\end{verbatim}
也可以通过下面的方式设置一个全局快捷键(这个快捷键的设置早在【引用】里就已经给出)

\begin{verbatim}
(global-set-key (kbd "C-c c") 'org-capture)
\end{verbatim}
\subsubsection{using capture 使用capture}
\label{sec:org8e3f7e3}
\begin{enumerate}
\item M-x org-capture
\end{enumerate}
执行org-capture.

\begin{enumerate}
\item C-c C-c
\end{enumerate}
返回捕获过程之前的窗口配置

\begin{enumerate}
\item C-c C-w
\end{enumerate}
定档(finalize)整个capture的过程,即将笔记移动到一个新的位置.

\begin{enumerate}
\item C-c C-k
\end{enumerate}
\subsubsection{{\bfseries\sffamily DONE} capture templates}
\label{sec:orgc256b6c}
 中途推出按钮.
这个地方并不是特别清楚.应该是定义模板的一种格式.设置模板的源代码为:
\begin{verbatim}
(setq org-capture-templates
      '(("t" "Todo" entry (file+headline "~/org/gtd.org" "Tasks")
	 "* TODO %?\n %i\n %a")
	("j" "Journal" entry (file+datetree "~/org/journal.org")
	 "* %?\n Enetered on %U\n %i\n %a")))
\end{verbatim}
其表达的意义是:
\begin{itemize}
\item 当使用t时便可以创建一个todo,并导出一个链接,链接的形式为:文件名+章节名,而后作为一个Tasks存储在\textasciitilde{}/org/gtd.org这个文档里.
\item \%?表示在把模板内容填充完毕之后,光标应该停留的位置;
\item \%i (initial content) 表示被填充的初始内容,只有在有文本内容被选中,且使用了C-u前缀进行capture的前提下这个功能才能使用.
\item \%a annotation,注释.通常是用org-store-link创建的链接
\end{itemize}

\subsection{refile and copy 文件重归档与复制}
\label{sec:org74de062}
本节的意思,似乎就是简化剪切,切换,粘贴这一整套的文本条目重新归档的过程.
\begin{enumerate}
\item C-c C-w
\end{enumerate}
C-c C-w 就是说,要把这一小节(光标所在的小节)的内容归档至其他的某个小节.

\begin{enumerate}
\item C-u C-c C-w
\end{enumerate}
使用refile界面跳转到标题.

\begin{enumerate}
\item C-u C-u C-c C-w
\end{enumerate}


\begin{enumerate}
\item C-c M-w
\end{enumerate}

\section{{\bfseries\sffamily INPROGRESS} agenda views}
\label{sec:org169a202}
Agenda是一种对零散的todo文件进行聚集处理的操作。

\subsection{agenda files}
\label{sec:org48cdc80}
\begin{enumerate}
\item C-c [ 将当前文件加入到agenda file列表中
\item C-c ] 将当前文件从agenda file列表中移除
\item C-'
\item C-, cycle through agenda file list, one after another
\end{enumerate}

\subsection{The Agenda Dispatcher 日程调度分配器}
\label{sec:org75b2417}
使用M-x org-agenda进行激活,或者使用快捷键C-c a.
分配器提供了以下一些默认的指令:
\begin{itemize}
\item a 创建一个日历形式的日程
\item t T 创建一个包含所有tudo项的列表
\item m M 创建一个匹配了表达式的所有headline的列表
\item s Create a list of entries selected by a boolean expression of keywords and/or regular expressions that must or must not occur in the entry. 不是特别理解这句话什么意思.
\end{itemize}
\subsection{The Weekly /Daily  Agenda}
\label{sec:org6c1a42a}
就像是传统的纸上的日程表一样,weekly-daily agenda给出每天或每周所需要干的事.
比如,在使用M-x org-agenda a命令时,其基本的思路是从org文件列表中提取条目信息编译形成当前周的日历.
\subsection{the global todo list 全局todo列表}
\label{sec:orgab97d41}
全局todo列表将所有的未完成的todo项目进行了一个统一的收集,可以用t关键字进行查询.
\begin{itemize}
\item M-x org-agenda t 展示全局todo列表
\item M-x org-agenda T 和一条相似,不过可以允许搜索特定的todo关键词
\end{itemize}
\subsection{Matching Tags and Properties 匹配标签和属性}
\label{sec:orgc4c92ca}
\section{{\bfseries\sffamily TODO} markup}
\label{sec:orgedb5766}

\section{{\bfseries\sffamily TODO} exporting}
\label{sec:org1e4b6d9}
这一章主要讨论如何使用org进行文档的导出.一般,关于文档导出的工作,可以通过C-c C-e进行调用.


\subsection{导出时需要的一些特殊信息}
\label{sec:orgce04601}
比如,可以在文档的所有位置(但是建议于开头处)插入此类:

\begin{verbatim}
#+TITLE: org基本笔记
\end{verbatim}

一般可供此类插入的信息主要包括:
\begin{itemize}
\item TITLE. 文章的名字
\item AUTHOR. 作者
\item DATE. 一个日期,或者org的时间戳(timestamp)
\item EMAIL. email
\item LANGUAGE. language code,如"en".
\end{itemize}

\subsection{table of contents 内容目录}
\label{sec:org90bdb5f}
在org中,导出会默认在第一个headline前面插入目录.可以通过下面的一些特殊的命令对目录进行自定义.

\begin{verbatim}
#+OPTIONS: toc:2          (only include two levels in TOC)
#+OPTIONS: toc:nil        (no default TOC at all)
\end{verbatim}

\subsection{include files 导入其他文件}
\label{sec:org44c9e7d}
可以在org文件里插入其他文件,比如,插入一段emacs的配置文件信息,将之作为src并以elisp的语法进行展示.
\begin{verbatim}
#+INCLUDE: "~/.emacs" src emacs-lisp
\end{verbatim}
一般,插入的文件的类型包括example, export, src这三种.

\subsection{comment lines 注释行}
\label{sec:org334183c}
注释符号为#号.

\subsection{正文开始:导出成不同格式的文件}
\label{sec:orge7e7789}

\subsubsection{ASCII UTF-8}
\label{sec:org8473597}
导出为txt文件.使用C-c C-e t a(scii) 或C-c C-e t u(tf-8) 

\subsubsection{HTML}
\label{sec:orgc53b1ec}
使用C-c C-e h h生成一个html文件,使用C-c C-e h o 生成并在浏览器里打开这样一个文件.

此处值得注意的是,org在进行文本转化时,将"<"与">"表达为"\&lt"与"\&gt".因此,如果要在org中插入一段原生的HTML代码,应当使用"",比如下面的例子:

\begin{verbatim}
@@html:<b>@@bold text@@html:</b>@@
\end{verbatim}
对于大范围的HTML代码块,可以通过下面的方法进行代码块的导出

\begin{verbatim}
#+HTML: Literal HTML code for export

#+BEGIN_EXPORT html
  All lines between these markers are exported literally
#+END_EXPORT
\end{verbatim}

\subsubsection{latex export}
\label{sec:org42fe6ed}
有关latex文本的导出,是一个很重要的地方.其重要之处在于,latex的语法比org复杂更多,因此,在这种转变的过程中,难免存在大量的部分是默认的.
下面将一一介绍如何把一个org文件转化为一个可编译的latex.
\begin{enumerate}
\item 设置document的class
\label{sec:org6108b6c}
org默认其为article类型,但是,当然,也可以自己定义所使用的latex的类,使用如下命令:
\begin{verbatim}
#+LATEX_CLASS: myclass
\end{verbatim}
当然,这样导入要求myclass必须在列表org-latex-classes里面.

\item 基本的导出命令.
\label{sec:orgb2a64a4}
\begin{enumerate}
\item C-c C-e l l 导出一个latex文件
\item C-c C-e l p 导出一个latex文件并将之转换为PDF.
\item C-c C-e l o 导出一个latex文件并将之转换为PDF,之后打开
\end{enumerate}
当然,需要强调的一个问题是,*上述方法均无法很好地处理latex中存在中文的问题(因为编译本质上用的是pdflatex而非xelatex)*
\item 在org中插入latex代码块
\label{sec:org8d4b283}
一般,org允许在文档中插入任意的latex代码块,其基本思路与HTML的插入类似,规则为:
\begin{itemize}
\item 行内插入.使用" any arbitrary LaTex Code"进行插入.
\item 单行插入.使用如下命令:
\begin{verbatim}
#+LATEX: any arbitrary LaTeX code
\end{verbatim}
\item 多行插入.使用:
\begin{verbatim}
#+BEGIN_EXPORT latex
  any arbitrary LaTeX code
#+END_EXPORT
\end{verbatim}
\end{itemize}
\end{enumerate}





\section{{\bfseries\sffamily TODO} publishing}
\label{sec:orgbbdf53c}

\section{{\bfseries\sffamily DONE} working with source code 在笔记里插入源码[\%]}
\label{sec:org7095523}
org 在编辑源码,运行源码,tangling源码与导出源码上都有一些贡献.
一般来说,一个源码都可以表现成下面的格式:
\begin{verbatim}
<body>
\end{verbatim}

其中
\begin{itemize}
\item ‘<name>’ is a string used to uniquely name the code block,
\item ‘<language>’ specifies the language of the code block, e.g., ‘emacs-lisp’, ‘shell’, ‘R’, ‘python’, etc.,
\item ‘<switches>’ can be used to control export of the code block,
\item ‘<header arguments>’ can be used to control many aspects of code block behavior as demonstrated below,
\item ‘<body>’ contains the actual source code.
\end{itemize}

通过C-c '进行代码块的编辑,但是常常的一串呢?都需要输入吗?不是这样的。
从\href{http://wenshanren.org/?p=327}{此处}找到了一个自定义的解决方案,我觉得或许可以.
首先,把下面的函数放入init文件中.


\begin{verbatim}
(defun org-insert-src-block (src-code-type)
  "Insert a `SRC-CODE-TYPE' type source code block in org-mode."
  (interactive
   (let ((src-code-types
	  '("emacs-lisp" "python" "C" "sh" "java" "js" "clojure" "C++" "css"
	    "calc" "asymptote" "dot" "gnuplot" "ledger" "lilypond" "mscgen"
	    "octave" "oz" "plantuml" "R" "sass" "screen" "sql" "awk" "ditaa"
	    "haskell" "latex" "lisp" "matlab" "ocaml" "org" "perl" "ruby"
	    "scheme" "sqlite")))
     (list (ido-completing-read "Source code type: " src-code-types))))
  (progn
    (newline-and-indent)
    (insert (format "#+BEGIN_SRC %s\n" src-code-type))
    (newline-and-indent)
    (insert "#+END_SRC\n")
    (previous-line 2)
    (org-edit-src-code)))

\end{verbatim}


之后,将下列快捷键绑定

\begin{verbatim}
(add-hook 'org-mode-hook '(lambda ()
			    ;; turn on flyspell-mode by default
			    (flyspell-mode 1)
			    ;; C-TAB for expanding
			    (local-set-key (kbd "C-<tab>")
					   'yas/expand-from-trigger-key)
			    ;; keybinding for editing source code blocks
			    (local-set-key (kbd "C-c s e")
					   'org-edit-src-code)
			    ;; keybinding for inserting code blocks
			    (local-set-key (kbd "C-c s i")
					   'org-insert-src-block)
			    ))
\end{verbatim}

之后,就可以通过C-c s i 快捷键插入一个代码块了。
此处参考\url{http://wenshanren.org/?p=327}的博客。

下面对几个特殊环节进行简要介绍.
这些内容均来自于\href{https://orgmode.org/guide/Working-with-Source-Code.html\#Working-with-Source-Code}{这里}.
\subsection{{\bfseries\sffamily TODO} using header arguments}
\label{sec:orgaa1b521}
\subsection{{\bfseries\sffamily TODO} evaluating code blocks}
\label{sec:org37dbf6b}
\subsection{{\bfseries\sffamily TODO} results of evaluation}
\label{sec:org4ee290f}
\subsection{{\bfseries\sffamily TODO} exporting code blocks}
\label{sec:org8a10c37}
\subsection{{\bfseries\sffamily TODO} extracting source code}
\label{sec:org00ae2e8}

\section{{\bfseries\sffamily TODO} miscellaneous}
\label{sec:org6fe31d1}
\end{document}